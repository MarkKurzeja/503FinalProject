% =====================================
% Purpose: Create a Robert Bringhurst style thesis paper using the 
% classicthesis package and some custom enhancements - this is 
% the default template for most of my documents
% =====================================

% =====================================
% Document Class and main packages
% =====================================

\documentclass[10pt,a4paper, hidelinks]{article} % KOMA-Script article scrartcl
\usepackage[nochapters, pdfspacing]{classicthesis} % [nochapters] [drafting] (puts date/time at bottom) [beramono] (changed mono spaced font)

% =====================================
% Packages in Use
% =====================================

%Math Packages
\usepackage{amsmath}
\usepackage{amsfonts}
\usepackage{amssymb}
\usepackage{nicefrac} % For typsetting inline fractions
\usepackage{mathtools} % For substack and mathclap (underbrace helper commands)

%% Typography enhancements
\usepackage{microtype} % For awesome typographical improvements
\usepackage{booktabs} % Pretty \begin{tabular}
\usepackage{multicol} % For pretty multi-columns enviroments
\usepackage{xspace} % For use in a command to ensure proper spacing
%\usepackage{geometry} %uncomment this is you want full page documentation
\usepackage{graphicx} % for allowing pictures
\usepackage{float} % For the purpose of adding \begin{figure} [H]
\usepackage{lipsum} % For adding filler text
\usepackage{wasysym} % For the \newmoon command for the legal blobs

% For commenting on incomplete or new items
\usepackage{todonotes} % \missingfigure{} is the best command

% For inclusion of PDFs in the document \includepdf[pages={3,{},8-11,15}]{example.pdf}
% This will include page 3, a blank page, 8,9,10,11, and 15
% you can set this to draft to get boxes or final to get the default which includes
% the pages
% Options:
% 	fitpaper: Use this to insert the page as is - otherwise items are scaled to page
% 	rotateoversize: Attempt to rotate oversized pages and scale them
\usepackage[final]{pdfpages}

% =====================================
% Graphics 
% =====================================

% For quick graphics insert -- Full Line --
\newcommand{\qpic}[2]{
\begin{figure}[H]
\centering
\includegraphics[width=1\linewidth]{./#1}
\caption{#2}
\label{fig:#1}
\end{figure}
}

% For quick graphics insert -- Normal Size --
\newcommand{\qpics}[2]{
\begin{figure}[H]
\centering
\includegraphics[width=0.7\linewidth]{./#1}
\caption{#2}
\label{fig:#1}
\end{figure}
}

% =====================================
% Custom Macros that make life easier
% =====================================

% Description Enviroment Item Helper Commands
\newcommand{\im}[1]{\item[#1] \xspace}
\newcommand{\imp}[1]{\item[(#1)] \xspace}

% Auto-commas for long nominal and dollar amounts
\RequirePackage{siunitx}
\newcommand{\commasep}[1]{\num[group-separator={,}]{#1}}
\newcommand{\money}[1]{\$\commasep{#1}}

% Borrowing from tufte, this is the \newthought command that is 
% often used to bring about the change from one subsubsection
% to another and is a good way to bring things up logically into smaller
% bites
\newcommand{\newthought}[1]{
\vspace{11pt} \noindent
\spacedlowsmallcaps{#1}
}

% Code for the fast creation of bullet lists
\newcommand{\qb}[1]{\begin{itemize} #1 \end{itemize}}

% Code to produce spaced small caps in real text
\newcommand{\mysmallcaps}[1]{\spacedlowsmallcaps{#1}\xspace}

% Legal blobbing for reminding oneself to include information there: [<circle>]
\newcommand{\legalblob}{\ensuremath{\left[\newmoon\right]}}

% Simple math macros for probablility and expectations that are very common for math homework
\newcommand{\p}{\ensuremath{\mathbb{P}}\xspace}
\newcommand{\expt}[1]{\ensuremath{\mathbb{E}}\left[#1\right]\xspace}
\newcommand{\myvec}[1]{\underset{\sim}{#1}}
\newcommand{\myexp}[1]{\exp\left( #1\right) }

% =====================================
% For handling code blocks and other text
% =====================================

% Code to handle inputting code segments in R
% To import code use: \lstinputlisting[language=R]{h1code.r}
% To add code in-line use \begin{lstlisting}[language = {}] 
% \lstinputlisting[language={}]{file.txt} for  unformatted code
\usepackage{listings} 
\lstset{language=R} 
\usepackage{color}
\definecolor{mygreen}{rgb}{0,0.6,0}
\definecolor{mygray}{rgb}{0.5,0.5,0.5}
\definecolor{mymauve}{rgb}{0.58,0,0.82}

\lstset{ %
  backgroundcolor=\color{white},   % choose the background color; you must add \usepackage{color} or \usepackage{xcolor}
  basicstyle=\footnotesize,        % the size of the fonts that are used for the code
  breakatwhitespace=false,         % sets if automatic breaks should only happen at whitespace
  breaklines=true,                 % sets automatic line breaking
  captionpos=b,                    % sets the caption-position to bottom
  commentstyle=\color{mygreen},    % comment style
  deletekeywords={...},            % if you want to delete keywords from the given language
  escapeinside={\%*}{*)},          % if you want to add LaTeX within your code
  extendedchars=true,              % lets you use non-ASCII characters; for 8-bits encodings only, does not work with UTF-8
  frame=single,	                   % adds a frame around the code
  keepspaces=true,                 % keeps spaces in text, useful for keeping indentation of code (possibly needs columns=flexible)
  keywordstyle=\color{blue},       % keyword style
  language=R,                 % the language of the code
  otherkeywords={*,...},            % if you want to add more keywords to the set
  numbers=left,                    % where to put the line-numbers; possible values are (none, left, right)
  numbersep=5pt,                   % how far the line-numbers are from the code
  numberstyle=\tiny\color{mygray}, % the style that is used for the line-numbers
  rulecolor=\color{black},         % if not set, the frame-color may be changed on line-breaks within not-black text (e.g. comments (green here))
  showspaces=false,                % show spaces everywhere adding particular underscores; it overrides 'showstringspaces'
  showstringspaces=false,          % underline spaces within strings only
  showtabs=false,                  % show tabs within strings adding particular underscores
  stepnumber=2,                    % the step between two line-numbers. If it's 1, each line will be numbered
  stringstyle=\color{mymauve},     % string literal style
  tabsize=2,	                   % sets default tabsize to 2 spaces
  title=\lstname                   % show the filename of files included with \lstinputlisting; also try caption instead of title
}

% =====================================
% Beginning the main document
% =====================================

\begin{document}
\pagestyle{plain} 
\title{\rmfamily\normalfont\spacedallcaps{Learning the Madness - NCAA March Madness Predictions}}
\author{\spacedlowsmallcaps{Andrei Kopelevich $\bullet$ Mark Kurzeja $\bullet$ Eli Schultz }}
\date{} % no date or \today if you want to insert a date

\maketitle

\begin{abstract}
	Every year, millions of people fill out March Madness brackets in attempt to predict the outcome of the NCAA men's basketball tournament. A big part of the draw is that the tournament is notoriously difficult to predict, and predicting a perfect bracket is so improbable that Warren Buffet, a US investor, annually offers \$1MM a year for life to any of his employees who can predict the exact outcome of the tournament. Many algorithms for predicting the tournament attempt to predict results at the individual game level, based on how teams match up with one another. We will instead focus on predicting how many rounds a team will progress in the tournament based on its performance over the course of the preceding season, with the goal of building stable predictions and successfully predicting tournament outcomes.
\end{abstract}

\tableofcontents
\newpage

%%%%%%%%%%%%%%%%%%%%%%%%%%%%%%%%%%%%%%%%%%%%%%%%%%%%%%%%%%%%%%%%%%%%%%%%%%%%%%%%
%                                                                              %
%                          NCAA Tournament Background                          %
%                                                                              %
%%%%%%%%%%%%%%%%%%%%%%%%%%%%%%%%%%%%%%%%%%%%%%%%%%%%%%%%%%%%%%%%%%%%%%%%%%%%%%%%
\section{Background}
After all regularly scheduled NCAA basketball games are concluded in mid-March, 68 teams are selected to participate in the NCAA tournament. Every game is single elimination, meaning that as soon as a team loses it is eliminated from the tournament. Eight of the 68 teams originally chosen have to compete in so-called play-in games, after which 64 teams remain. We will follow the same framework as that followed by most bracket prediction contests and concern ourselves here with predicting how teams will perform once the field has been narrowed down to 64 contenders. At this point there are $2^{32} \cdot 2^{16} \cdot 2^8 \cdot 2^4 \cdot 2^2 \cdot 2 = 2^{63}$ possible outcomes, which gives one an idea of the scope of this challenge. Predicting a perfect bracket may be impossible, but we will see how close we can get.

The first step in our analysis was gathering data. There are many decades of team performance data available on the Sports Reference college basketball website free of charge. However, in recent years a number of more advanced metrics have been devised that have been shown to more accurately capture a team's strength. Since this data is only available starting from 1993, we limited our analysis to that timeframe. Our working data set ultimately consisted of 12 continuous quantitative predictors, and one categorical response corresponding to the number of victories a team had in the tournament (ranging from 0 for a team that lost immediately to 6 for a national champion).

The predictors can be thought of as belonging to six broad buckets corresponding to aspects of a team performance:

\begin{itemize}
	\item Winning percentage, strength of schedule (SOS) and simple rating system (SRS) are all holistic measures.
	\item Free throw attempt rate (ft\_rate0), free throw attempts per field goal attempt (fta\_per\_fga\_pct), and three-point attempts per field goal attempt (fg3a\_per\_fga\_pct), and assist rate (ast\_pct) all help classify a team's offensive playing style. The idea is that three-point shots are high risk and high reward while free throws are low risk and low reward. Thus teams that attempt more three point shots may have more variable performances from game to game, while teams that attempt more free throws may be more consistent. Teams with higher assist rates should be expected to play more collaboratively, while teams with low assist rates rely more on strong individual performances.
	\item Total rebound percentage (trb\_pct) captures how well a team recovers the ball following shots missed by either themselves or their opponents.
	\item True shooting percentage (ts\_pct) and effective field goal percentage (efg\_pct) both capture how well a team shoots the ball.
	\item Turnover percentage (tov\_pct) captures how well a team is able to avoid allowing the other team to take the ball away (better teams should generally have lower turnover percentages).
	\item Block percentage (blk\_pct)  captures how well a team is able to block the other team's shots.
\end{itemize}

It is worth noting that the vast majority of these statistics capture how well a team performs offensively.  

%%%%%%%%%%%%%%%%%%%%%%%%%%%%%%%%%%%%%%%%%%%%%%%%%%%%%%%%%%%%%%%%%%%%%%%%%%%%%%%%
%                                                                              %
%                            The prediction problem                            %
%                                                                              %
%%%%%%%%%%%%%%%%%%%%%%%%%%%%%%%%%%%%%%%%%%%%%%%%%%%%%%%%%%%%%%%%%%%%%%%%%%%%%%%%
\section{Preliminary Data Exploration}
We began by performing data exploration and visualization, as well as dimension reduction, to inform our analysis.

Summaries of the predictors are provided below.

\missingfigure{SUMMARY}

At a high level, the first three predictors provide a good sanity check in that they suggest that the teams in the tournament are generally better than average teams. The first quartile for winning percentage (percentage of games that a given team won) is well above 50\% for the teams in our data set. Our teams also generally have strength of schedule and simple rating system scores well above zero, which is to be expected given that those metrics are designed such that an average team will have a score around zero. It may seem a bit strange at first that stronger teams will tend to have a better SOS score, but this can be explained by the fact that strong teams tend to be in the same conferences as other strong teams, and therefore play other strong teams more frequently.

For model selection and interpretation purposes, it is also worth noting that a number of the predictors have unusually large and/or small outliers.

A pairwise scatterplot of the predictors is provided below.

\missingfigure{GGPAIRS--Most important part of report!}

Many of the correlations are relatively small. Unsurprisingly, in some cases where two predictors belong to the same bucket (per the framework above) we notice very strong correlations. Effective field goal percentage and true shooting percentage are the most strongly correlated predictors ($r \approx 0.96$), which is to be expected given that both measure how well a team shoots the ball. Free throw rate and free throw attempts per field goal attempt ($r \approx 0.922$) both capture how often a team tends to shoot free throws in slightly different ways. Finally, the strong correlations between SRS and wining percentage ($r \approx 0.518$) and SRS and SOS ($r \approx 0.867$) reflect the fact that SRS is meant to capture team performance, adjusted for strength of schedule. 

However, there are a few other correlations that are much weaker, but perhaps a little more interesting conceptually. For instance, there is a weak negative correlation between rebounding percentage and free throw attempts per field goal attempt ($r \approx -0.311$), a correlation which does not seem to have an obvious intuitive explanation based on the buckets outlined above. Similarly, both true shooting 

The data projected onto the first two principal components, which only explain about 44\% of the variability within the data set, appear as follows:

\missingfigure{Principal components}

While we do not necessarily observe perfect clustering corresponding to each of the possible win totals, it is encouraging to see that the first principal component in particular seems to do a fairly good job of differentiating weak teams from strong ones, with teams that advanced further in the tournament tending to have lower scores for that component. Although this trend is much less pronounced, it appears that stronger teams may also tend to have slightly higher scores for the second principal component. 

%%%%%%%%%%%%%%%%%%%%%%%%%%%%%%%%%%%%%%%%%%%%%%%%%%%%%%%%%%%%%%%%%%%%%%%%%%%%%%%%
%                                                                              %
%                              Prelim Discussion                               %
%                                                                              %
%%%%%%%%%%%%%%%%%%%%%%%%%%%%%%%%%%%%%%%%%%%%%%%%%%%%%%%%%%%%%%%%%%%%%%%%%%%%%%%%
\section{Preliminary Discussion of Methods}

\subsection{Methods Used}
\lipsum
\subsection{Discussion of Stacking}
\lipsum
\subsection{What data is available?}
\lipsum
%%%%%%%%%%%%%%%%%%%%%%%%%%%%%%%%%%%%%%%%%%%%%%%%%%%%%%%%%%%%%%%%%%%%%%%%%%%%%%%%
%                                                                              %
%                               Impl. Discussion                               %
%                                                                              %
%%%%%%%%%%%%%%%%%%%%%%%%%%%%%%%%%%%%%%%%%%%%%%%%%%%%%%%%%%%%%%%%%%%%%%%%%%%%%%%%
\section{Discussion of Implementation}
\lipsum

%%%%%%%%%%%%%%%%%%%%%%%%%%%%%%%%%%%%%%%%%%%%%%%%%%%%%%%%%%%%%%%%%%%%%%%%%%%%%%%%
%                                                                              %
%                                Model Results                                 %
%                                                                              %
%%%%%%%%%%%%%%%%%%%%%%%%%%%%%%%%%%%%%%%%%%%%%%%%%%%%%%%%%%%%%%%%%%%%%%%%%%%%%%%%
\section{Model Results}
\lipsum

%%%%%%%%%%%%%%%%%%%%%%%%%%%%%%%%%%%%%%%%%%%%%%%%%%%%%%%%%%%%%%%%%%%%%%%%%%%%%%%%
%                                                                              %
%                           Conclusion and follow up                           %
%                                                                              %
%%%%%%%%%%%%%%%%%%%%%%%%%%%%%%%%%%%%%%%%%%%%%%%%%%%%%%%%%%%%%%%%%%%%%%%%%%%%%%%%
\section{Conclusion and Further Research}
\lipsum


\end{document}
